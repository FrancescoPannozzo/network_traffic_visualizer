% !TeX encoding = UTF-8
% !TeX program = pdflatex
% !TeX spellcheck = it_IT
\documentclass[binding=0.6cm]{sapthesis}
\usepackage{microtype}
\usepackage[italian]{babel}
\usepackage[utf8]{inputenc}
\usepackage{hyperref}
\hypersetup{pdftitle={La mia tesi},pdfauthor={Francesco Biccari}}
\title{Visualizzazione e simulazione grafica di traffico di reti}
\author{Francesco Pannozzo}
\IDnumber{699427}
\course{Laurea Triennale in Informatica}
\courseorganizer{Facoltà di Ingegneria dell'Informazione, Informatica e Statistica}
\AcademicYear{2023/2024}
\advisor{Prof. Daniele De Sensi}
\authoremail{francesco.pannozzo@libero.it}
\copyyear{2024}
\thesistype{Relazione di tirocinio}
\begin{document}
\frontmatter
\maketitle
\dedication{Dedicato alla\\ mia famiglia}
\begin{abstract}
Questa relazione descrive il lavoro di tirocinio interno svolto presso l'università La Sapienza, 
concretizzato nella realizzazione 
di un progetto volto a realizzare un software per poter visualizzare in forma grafica l'andamento 
del traffico di una rete.
Il progetto ha come obiettivo di mostrare il traffico di rete al variare del tempo e ciò viene raggiunto
tramite grafiche e animazioni generate programmaticamente. L'idea dell'ambito di tirocinio nasce 
dalla volontà di sperimentare una realizzazione front-end tramite la libreria Manim, un motore di animazioni per video
matematici esplicativi..

\end{abstract}
\tableofcontents
\mainmatter
\chapter{Introduzione}
...
\chapter{Progettazione}
...
\begin{equation}
(l+1)(m+1)>ml
\end{equation}
\backmatter
\cleardoublepage
\phantomsection % Give this command only if hyperref is loaded
\addcontentsline{toc}{chapter}{\bibname}
% Here put the code for the bibliography. You can use BibTeX or
% the BibLaTeX package or the simple environment thebibliography.
\end{document}